
\chapter{Conclusions}
\label{cha:conclusions}

After completing the coding section of the project it has been possible to evaluate how well the whole experience went.


\section{Strengths}
\label{sec: strengths}
First of all the project was a success; we were able to implement a multi-player space shooter type game within the time limit specified in the module.\\\\
The game features several items that we did not originally think we would be able to implement, these features are the background music which was added towards the end of the project, graphics for the units and projectiles which were implemented with a few weeks to go and also the boss enemies which add a new level of difficulty to the game making it more interesting.\\\\
The management side of the project was also successful as the team were very good at communicating ideas between each other. To do this the team organised weekly face-to-face meetings to openly discuss the week's work as well as evaluate what had taken place that week and what still needs to be done. Although the project may have seen to be rushed towards the end, it was still delivered on time and is playable both as a single player game but also as a multi-player networked game.\\\\
During the development of the project the team faced a setback where one member of the team had to leave the UK, despite this inconvenience the team still managed to communicate effectively through email and Facebook and didn't cause too much disruption to the project as a whole. In this scenario the team decided it would be best to break the project up into larger sections so the team member could work alone without needing as much communication with the rest of the time like the start of the project had done. This was a success and the team member managed to contribute both to the coding sections of the game as well as the final report.
\section{Limitations}
\label{sec: limitations}
Some of the game's limitations have been documented in the testing and validation section.\\\\
Although the game itself functions properly and is playable there are some slight bugs as with any software project. After playing the game for a few minutes, the number of enemies and subsequently projectiles increases. The game manages to prune the arrays containing the objects when they leave the screen but soon the game starts to slow down and become slightly unresponsive. This bug was only introduced in the last week and due to time limitations, still exists.\\\\
The project did become delayed towards the end of the project, and, although it wasn't too much of an issue there are some claims of it being 'hacked together'. No one in the team originally had any experience in network and sockets programming which was required for the game, it was decided that this part would be left towards the end of the project while we would learn more during the term through lectures and other modules. The networking section of the project does work properly, however the code is quite messy and some shortcuts have been taken. If time was not a factor this section could have been smoother and implemented better.\\\\
As previously mentioned, one of the team members had to leave the UK during the development of the project. Communication became harder as it was only possible to communicate through electronic forms such as email and Facebook rather than the preferred face-to-face meetings. It also mean't that the team member was not present for the final demonstration to explain their part of the code. Despite this setback the team still managed to proceed with the project and the team member mentioned was able to contribute to all aspects of the project (coding and report) successfully.\\\\
There are also some other limitations that relate to the game play. The enemies all have the same path to move along (except for the boss enemies) which makes the game slightly predictable. There are other paths available to use but not method to call them, perhaps a random way of choosing a path should be implemented. Also, the mouse movement can be quite sensitive if the user is not used to it which makes collisions with enemies hard to avoid.
\section{Continuation}
\label{sec: continuation}
If the project were to be continued without a time limitation then the game could be improved. The networking side of the game could be more stable and tidy and as the project is already quite abstract perhaps certain features of the current existing game can be improved, for example the enemy paths.\\\\
During the project one team member had to be away from the UK which led to certain communication issues. If the project had no time limit the team may have decided to give that person a break until a time where they could become more involved. However, in this case the team was able to work around this problem and being forced to overcome the communication difficulties which occurred with a relative degree of ease.\\\\
Outlined in the requirements of this document is a list of future requirements which could be implemented if the team were to continue the project. However, this list isn't too long which identifies that the scope of this project isn't very large. Currently the only things the player can do is move and shoot the enemies, or to do the same thing along side a network player. The game could be improved though to become more exciting but the core principal of the game isn't very large.\\\\
Currently when the player 'dies' the game ends and displays a 'Game Over' screen (including in network play). To play the game again the user has to relaunch the program, future versions of the game could see the possibility for an endless mode (as documented in future requirements) where the user is re-spawned after they 'die' as well as having the option of a 'Play Again' button to avoid relaunching the program.
\section{Future Projects}
\label{sec: future}
From completing this project the team has learnt a lot about the processes of software engineering and development and would make changes to the way in which the software was developed in the future.\\\\
The software management aspect of the game seemed to run smoothly but actually there was more work to do than what had first been anticipated. Getting to the first version of the software (v0.1) took quite long (a few weeks), this seemed to be acceptable as it would include the core basics and would see the implementation of the base cases. It was then thought that the later sections of code would be able to be introduced without changing the code too much. One of these examples is with the networking side of the program. As mentioned previously the team had relatively little experience in sockets programming and it was decided that this section would be left until last to implement. Looking back, this was perhaps the wrong decision. The thinking behind leaving the networking until later was that the team could learn more during the term from other modules as well as programming lectures. In the future if the team had little knowledge of an important aspect of the project it would be advisable to research that field from the start and try to implement it in small steps rather than as a whole when time is running out.\\\\
The scope of the project is quite small as there is little else the user can do rather than move and shoot. Although the game is large enough for the purpose of this assignment, in future projects perhaps a different game type could be suggested. Conversely, the team all enjoyed working together to produce the game and the course has benefitted each individual for future projects. The team believes the game works well despite having some small issues but all team members are happy with the output of both the software and the report.